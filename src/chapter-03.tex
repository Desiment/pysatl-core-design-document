\chapter{Заинтересованные лица}

Настоящий раздел описывает различные типы заинтересованных сторон, которым будет интересен данный документ, а также их потенциальные опасения/запросы связанные с дизайном ядра. 

Мы отметим, что представленные здесь стороны/лица необязательно относятся к пользователям, которые взаимодействуют с системой.

\noindent\rule{\textwidth}{0.5pt}
\section{Разработчики ядра PySATL}

Это лица непосредственно принимающие участие в написании кода для ядра. Для них приоритетным является ряд возможностей:
\begin{itemizecmp}
    \item Добавление новой функциональности или оптимизация уже существующей;
    \item Покрытие кодовой базы тестами; в частности регрессионными тестами;
    \item Сопровождение существующей кодовой базы;
    \item Расширение коллектива разработчиков;
\end{itemizecmp}

\textbf{1. Добавление новой функциональности} \ Разработчики ядра расширяют и модифицируют библиотеку в контексте следующих задач:
\begin{itemizecmp}
    \item Добавление новых параметрических семейств/новых операций над распределениями;
    \item Добавление новых числовых и/или функциональных характеристик распределений;
    \item Добавление новых численных методов для вычисления характеристик распределений;
    \item Добавление новых численных методов для вычисления операций над распределениями;
\end{itemizecmp}

\textbf{2. Тестирование} \ Помимо расширения функциональности, разработчики ядра покрывают кодовую базу тестами; Им необходимо иметь ряд механизмов для:
\begin{itemizecmp}
    \item автоматического тестирования новой функциональности, касающейся свойств добавляемых распределений/семейств/операций 
    \item создания регрессионных тестов для функциональности, оперирующей с псевдо-случайным данными
\end{itemizecmp}

\begin{example}
При создании нового семейств распределений разработчик указывает что все его представители имеют носитель на отрезке $[0; 1]$. Это свойство распределения которое можно проверить например сгенерировав выборку из данного распределения и проверив что все элементы выборки лежат в этом отрезке.
\end{example}

%Тесты кода, вычисляющего оценки параметров распределений или проверяющего статистический критерий, обычно устроены следующим образом. Генерируется  выборка из распределения, над ней выполняется необходимая статистическая процедура, и результаты сравниваются с ожидаемыми. При таком подходе неизбежно что тесты периодечески будут не проходить.

\begin{example}
Валидационные примеры использования статистических процедур часто служат основной для регрессионых тестов этих самых процедур. Библиотека должна обеспечивать воспроизводимость генерации выборок.
\end{example}

\textbf{3. Сопровождение кодовой базы и расширения коллектива разработчиков}\\
Сопровждение кодовой базы и расширенние коллектива разработчиков осуществляется засчет нескольких инструментов
\begin{itemizecmp}
    \item Архитектурная документация; настоящий документ является её частью. В частности, так как ядро планируется активно использовать в других библиотеках, необходимо явно отразить какие части системы являются публично доступными
    \item Техническая и пользовательская документация; наличие качественной документации, содержащей примеры использования и описывающей роли компонент сисемы
\end{itemizecmp}

\noindent\rule{\textwidth}{0.5pt}
\section{Разработчики других библиотек в PySATL}

К этой категории относятся все разработчики PySATL, которые либо не принимают непосредственного участия в разработке ядра, либо делают это эпизодически, добавляя функциональность под конкретные нужды проектов (в последнем случае они относятся к первой категории заинтерсованных лиц). Их основные интересы по отоншению к ядру состоят в следующем.
\begin{enumeratecmp}
    \item Простота интеграции ядра в другие библиотеки PySATL;
    \item Конфигурируемость ядра для типичных сценариев использования; 
\end{enumeratecmp}

%#todo[ММ][На самом деле каждая библиотека является отдельным стейкхолдером; Так experiment нуждается в выше указанном конфигурировании семплинга, а mpest имеет совершенно другие запросы. nmvm это вообще отдельная головная боль]

\textbf{1. Простота интеграции ядра в другие библиотеки PySATL}\\
Сейчас PySATL активно использует связку Numpy/SciPy для большинства математических вычислений; Это означает что при переходе на ядро, во всех местах использующих эту связку, замена не должна вызвать осложнений. В частности ядро 
\begin{itemizecmp}
    \item Не должно обязывать разработчика других библиотек к конфигурации численных методов, использующихся в ядре;
    \item Должно иметь всю ту же функциональность, что и модуль \texttt{statistics} библиотеки SciPy;
    \item Должно предоставлять унифицированный интерфейс для прочих математических функций;
\end{itemizecmp}

Здесь надо пояснить что такое прочие математические функции. При реализации различных оценок, иногда вознкиают довольно нетривиальные объекты (так pysatl-nmvm использует многочлены Белла для вычислений, а mpest - различные методы оптимизации). Так как существует множество математических библиотек, которые могут предоставлять данные возможности, то:
\begin{enumeratecmp}
    \item Возможно, конечный пользователь библиотеки (не обязательно ядра) захочет чтобы использовались какие-то конкретные математические пакеты (что очень может быть в случае интегрирования/оптимизации)
    \item Наличие единного интерфейса для математических утилит не приведет к тому что внутри разных библиотек (или даже одной библиотеки) используются разные математические пакеты 
\end{enumeratecmp}

К тому же, многие из математических пакетов, предоставляющих редкие, но нужные функции, являются либо проприетарными, либо с вирусными лицензиями. Использование и тех, и других, не подходит под лицензионные ограничения; создание единого интерфейса для математических утилит позволяет распостранять PySATL под MIT лицензией, используя вирусную лицензию только для реализации этих интерфейсов. Это значительно позволит сократить ресурсы на прототипирование разлчиных библиотек на базе ядра.

\textbf{2.  Конфигурируемость ядра для типичных сценариев использования}\\

\noindent\rule{\textwidth}{0.5pt}
\section{Руководители проекта PySATL}

\begin{itemizecmp}
    \item Представление о планах разработки ядра и его функциональности на каждом этапе
    \item Гарантии предоставляемые ядром
\end{itemizecmp}

\noindent\rule{\textwidth}{0.5pt}
\section{Инженеры-исследователи}
Программисты, которым нужны только артефакты математической статитистики.  Математики, статистики и люди занимающиеся численными методами.

\begin{itemizecmp}
    \item Поддержка сложных операций над распределениями, смесями и т.п.
    \item Бенчмаркинг новых алгоритмов
\end{itemizecmp}