\chapter{Заинтересованные лица}

Настоящий раздел описывает различные типы заинтересованных сторон, которым будет интересен данный документ, а также их потенциальные опасения/запросы связанные с дизайном ядра. 

Мы отметим, что представленные здесь стороны/лица необязательно относятся к пользователям, которые взаимодействуют с системой.

\noindent\rule{\textwidth}{0.5pt}
\section{Разработчики ядра PySATL}

Это лица непосредственно принимающие участие в написании кода для ядра. Для них приоритетным является ряд возможностей:
\begin{itemizecmp}
    \item Добавление новой функциональности или оптимизация уже существующей;
    \item Покрытие кодовой базы тестами; в частности регрессионными тестами;
    \item Сопровождение существующей кодовой базы;
    \item Расширение коллектива разработчиков;
\end{itemizecmp}

\textbf{1. Добавление новой функциональности} \ Разработчики ядра расширяют и модифицируют библиотеку в контексте следующих задач:
\begin{itemizecmp}
    \item Добавление новых параметрических семейств/новых операций над распределениями;
    \item Добавление новых числовых и/или функциональных характеристик распределений;
    \item Добавление новых численных методов для вычисления характеристик распределений;
    \item Добавление новых численных методов для вычисления операций над распределениями;
\end{itemizecmp}

\textbf{2. Тестирование} \ Помимо расширения функциональности, разработчики ядра покрывают кодовую базу тестами; Им необходимо иметь ряд механизмов для:
\begin{itemizecmp}
    \item автоматического тестирования новой функциональности, касающейся свойств добавляемых распределений/семейств/операций 
    \item создания регрессионных тестов для функциональности, оперирующей с псевдо-случайным данными
\end{itemizecmp}

\begin{example}
При создании нового семейств распределений разработчик указывает что все его представители имеют носитель на отрезке $[0; 1]$. Это свойство распределения которое можно проверить например сгенерировав выборку из данного распределения и проверив что все элементы выборки лежат в этом отрезке.
\end{example}

%Тесты кода, вычисляющего оценки параметров распределений или проверяющего статистический критерий, обычно устроены следующим образом. Генерируется  выборка из распределения, над ней выполняется необходимая статистическая процедура, и результаты сравниваются с ожидаемыми. При таком подходе неизбежно что тесты периодечески будут не проходить.

\begin{example}
Валидационные примеры использования статистических процедур часто служат основной для регрессионых тестов этих самых процедур. Библиотека должна обеспечивать воспроизводимость генерации выборок.
\end{example}

\textbf{3. Сопровождение кодовой базы и расширения коллектива разработчиков}\\
Сопровждение кодовой базы и расширенние коллектива разработчиков осуществляется засчет нескольких инструментов
\begin{itemizecmp}
    \item Архитектурная документация; настоящий документ является её частью. В частности, так как ядро планируется активно использовать в других библиотеках, необходимо явно отразить какие части системы являются публично доступными
    \item Техническая и пользовательская документация; наличие качественной документации, содержащей примеры использования и описывающей роли компонент сисемы
\end{itemizecmp}

\noindent\rule{\textwidth}{0.5pt}
\section{Разработчики других библиотек в PySATL}

К этой категории относятся все разработчики PySATL, которые либо не принимают непосредственного участия в разработке ядра, либо делают это эпизодически, добавляя функциональность под конкретные нужды проектов (в последнем случае они относятся к первой категории заинтерсованных лиц). Их основные интересы по отоншению к ядру состоят в следующем.
\begin{enumeratecmp}
    \item Простота интеграции ядра в другие библиотеки PySATL;
    \item Конфигурируемость ядра для типичных сценариев использования; 
\end{enumeratecmp}

%#todo[ММ][На самом деле каждая библиотека является отдельным стейкхолдером; Так experiment нуждается в выше указанном конфигурировании семплинга, а mpest имеет совершенно другие запросы. nmvm это вообще отдельная головная боль]

\textbf{1. Простота интеграции ядра в другие библиотеки PySATL}\\
Сейчас PySATL активно использует связку Numpy/SciPy для большинства математических вычислений; Это означает что при переходе на ядро, во всех местах использующих эту связку, замена не должна вызвать осложнений. В частности ядро 
\begin{itemizecmp}
    \item Не должно обязывать разработчика других библиотек к конфигурации численных методов, использующихся в ядре;
    \item Должно иметь всю ту же функциональность, что и модуль \texttt{statistics} библиотеки SciPy;
    \item Должно предоставлять унифицированный интерфейс для прочих математических функций;
\end{itemizecmp}

Здесь надо пояснить что такое прочие математические функции. При реализации различных оценок, иногда вознкиают довольно нетривиальные объекты (так pysatl-nmvm использует многочлены Белла для вычислений, а mpest - различные методы оптимизации). Так как существует множество математических библиотек, которые могут предоставлять данные возможности, то:
\begin{enumeratecmp}
    \item Возможно, конечный пользователь библиотеки (не обязательно ядра) захочет чтобы использовались какие-то конкретные математические пакеты (что очень может быть в случае интегрирования/оптимизации)
    \item Наличие единного интерфейса для математических утилит не приведет к тому что внутри разных библиотек (или даже одной библиотеки) используются разные математические пакеты 
\end{enumeratecmp}

К тому же, многие из математических пакетов, предоставляющих редкие, но нужные функции, являются либо проприетарными, либо с вирусными лицензиями. Использование и тех, и других, не подходит под лицензионные ограничения; создание единого интерфейса для математических утилит позволяет распостранять PySATL под MIT лицензией, используя вирусную лицензию только для реализации этих интерфейсов. Это значительно позволит сократить ресурсы на прототипирование разлчиных библиотек на базе ядра.

\textbf{2.  Конфигурируемость ядра для типичных сценариев использования}\\
Различные библиотеки проекта PySATL предъявляют различные, зачастую противоречивые, требования к вычислительным характеристикам ядра. Библиотека должна предоставлять механизмы тонкой настройки без необходимости модификации её исходного кода. Это включает в себя:

\begin{itemize}[noitemsep, topsep=0pt, parsep=0pt]
    \item \textbf{Выбор численных методов:} Возможность выбора конкретной реализации алгоритма (например, для вычисления интеграла или оптимизации) в зависимости от требований к точности, скорости или устойчивости. Библиотека \texttt{experiment} может требовать максимальной скорости для генерации больших выборок, в то время как \texttt{mpest} — максимальной надёжности и точности для сходимости алгоритмов оценки параметров.
    \item \textbf{Управление вычислительными ресурсами:} Настройка параметров, влияющих на производительность и использование памяти. Например:
    \begin{itemizecmp}
        \item Установка допусков (tolerances) для итеративных алгоритмов;
        \item Задание лимитов на количество итераций;
        \item Управление размером кэшей для повторных вычислений.
    \end{itemizecmp}
    \item \textbf{Стратегии семплирования:} Конфигурация генераторов псевдослучайных чисел (ГПСЧ), включая:
    \begin{itemizecmp}
        \item Выбор конкретного алгоритма ГПСЧ (например, Mersenne Twister, PCG64);
        \item Установка начального значения (seed) для обеспечения воспроизводимости результатов;
        \item Управление параллельными потоками генерации для эффективного использования многоядерных систем.
    \end{itemizecmp}
    \item \textbf{Политики обработки ошибок:} Возможность настройки реакции на исключительные ситуации --- от строгого прерывания вычисления с выводом подробной ошибки до возврата специальных значений (NaN, inf) или использования fallback-алгоритмов для обеспечения отказоустойчивости.
    \item \textbf{Абстракция математического бэкенда:} Единый интерфейс для низкоуровневых математических операций (линейная алгебра, специальные функции, оптимизация), позволяющий подменять реализации без изменения кода, зависящего от ядра. Это гарантирует, что библиотеки \texttt{nmvm} и \texttt{mpest} будут использовать согласованный computational stack, конфигурируемый централизованно.
\end{itemize}


\noindent\rule{\textwidth}{0.5pt}
\section{Руководители проекта PySATL}

Данная категория включает лиц, ответственных за планирование развития проекта в целом, распределение ресурсов и координацию между рабочими группами. Их интересы сфокусированы на стратегических аспектах разработки и эксплуатации ядра, а также на его интеграции в общую экосистему PySATL.

\begin{itemizecmp}
    \item Представление о планах разработки ядра и его функциональности на каждом этапе;
    \item Гарантии, предоставляемые ядром;
    \item Эффективное использование ресурсов и соблюдение сроков;
    \item Минимизация рисков, связанных с техническим долгом и зависимостями.
\end{itemizecmp}

\textbf{1. Планы разработки и функциональность} \\
Руководителям проекта необходимо иметь чёткое и актуальное представление о состоянии и roadmap ядра для принятия обоснованных решений. Это включает:
\begin{itemizecmp}
    \item Соответствие фактической функциональности ядра заявленным планам и требованиям других подсистем PySATL;
    \item Наличие документации, достаточной для оценки готовности компонент к интеграции;
    \item Понятный и измеримый прогресс разработки, позволяющий оценивать выполнение этапов.
\end{itemizecmp}

\textbf{2. Гарантии, предоставляемые ядром} \\
Поскольку ядро является фундаментальным компонентом всего проекта, от его качества и стабильности зависит работа всех надстроенных над ним библиотек. Руководители заинтересованы в наличии гарантий, обеспечиваемых за счёт:
\begin{itemizecmp}
    \item Всестороннего тестирования, включая модульные, интеграционные и регрессионные тесты;
    \item Чёткого определения и поддержания публичных интерфейсов (API), что обеспечивает стабильность для потребителей ядра;
    \item Предсказуемой политики управления версиями, минимизирующей обратно несовместимые изменения;
    \item Исчерпывающей документации, снижающей риски misinterpretation API и его неправильного использования.
\end{itemizecmp}

\textbf{3. Эффективность использования ресурсов и сроки} \\
Процесс разработки ядра должен быть организован таким образом, чтобы:
\begin{itemizecmp}
    \item Архитектурные решения способствовали повторному использованию кода и сокращению избыточности;
    \item Существовали механизмы для быстрого прототипирования и валидации новых идей без угрозы нарушения стабильности основной codebase;
    \item Соблюдались согласованные сроки поставки критически важной для других подсистем функциональности.
\end{itemizecmp}

\textbf{4. Управление рисками} \\
Ключевым аспектом является проактивное выявление и снижение потенциальных рисков, в частности:
\begin{itemizecmp}
    \item Контроль над внешними зависимостями, их лицензионной чистотой и совместимостью;
    \item Минимизация накопления технического долга за счёт регулярного рефакторинга и применения современных практик разработки;
    \item Наличие плана по обеспечению долгосрочной поддерживаемости и расширяемости кодовой базы.
\end{itemizecmp}

\noindent\rule{\textwidth}{0.5pt}
\section{Инженеры-исследователи}

К данной категории относятся специалисты, применяющие ядро PySATL для решения прикладных задач: программисты, математики, статистики и исследователи в области численных методов. Они используют систему как инструмент для реализации и проверки научных гипотез, прототипирования алгоритмов и проведения вычислительных экспериментов.

\begin{itemizecmp}
    \item Поддержка сложных операций над распределениями, смесями и иерархическими моделями;
    \item Бенчмаркинг новых алгоритмов и сравнение с существующими аналогами;
    \item Гибкость и выразительность для описания нестандартных вероятностных моделей;
    \item Воспроизводимость и контролируемость вычислительных экспериментов.
\end{itemizecmp}

\textbf{1. Поддержка сложных операций и моделей} \\
Инженеры-исследователи работают с комплексными вероятностными структурами, выходящими за рамки стандартных распределений. Их интерес заключается в возможности:
\begin{itemizecmp}
    \item Конструировать составные распределения (смеси, свёртки, произведения);
    \item Определять иерархические (многоуровневые) вероятностные модели;
    \item Применять нелинейные функциональные преобразования к распределениям;
    \item Работать с усечёнными, цензурированными или модифицированными распределениями.
\end{itemizecmp}

\textbf{2. Бенчмаркинг и валидация алгоритмов} \\
Для данной аудитории критически важна возможность оценивать эффективность и корректность реализуемых методов:
\begin{itemizecmp}
    \item Наличие встроенных средств для замеров производительности (профилирования) отдельных операций;
    \item Возможность сравнивать численные результаты, полученные с помощью ядра, с эталонными реализациями или аналитическими выкладками;
    \item Доступ к низкоуровневым параметрам численных методов (таким как точность, допуски, критерии остановки) для тонкой настройки и исследования сходимости алгоритмов.
\end{itemizecmp}

\begin{example}
Исследователь, разрабатывающий новый метод оценки параметров, должен иметь возможность сгенерировать набор синтетических данных из известного распределения, применить свой метод, сравнить полученные оценки с истинными значениями параметров и измерить время выполнения алгоритма, обеспечивая при этом полную воспроизводимость эксперимента.
\end{example}

\textbf{3. Гибкость и выразительность} \\
Система должна предоставлять достаточный уровень абстракции для лаконичного и интуитивно понятного выражения математических концепций:
\begin{itemizecmp}
    \item Декларативный способ задания моделей, приближенный к математической нотации;
    \item Возможность легко комбинировать примитивы ядра для создания специализированных конструкций без необходимости написания большого объема boilerplate-кода;
    \item Доступ к символьным представлениям преобразований для анализа и оптимизации выражений до их численного вычисления.
\end{itemizecmp}

\textbf{4. Воспроизводимость и контролируемость} \\
Так как работа инженера-исследователя носит экспериментальный характер, ядро должно предоставлять инструменты для обеспечения строгости научного процесса:
\begin{itemizecmp}
    \item Детерминированная работа генератора псевдослучайных чисел с возможностью сохранения и восстановления состояния (seed);
    \item Логирование ключевых этапов вычислений и принятых решений (например, в алгоритмах оптимизации);
    \item Чёткое разделение между абстрактной спецификацией модели и конкретным численным методом её вычисления.
\end{itemizecmp}