
%%%%
%% Foundations: Logic, set theory, topology %%
%%%%

%%%% Logic
\newmathcommand{\excludeor}{\mathbin{\star}}

%%%% Set theory
\DeclarePairedSplitDelimiter{\set}{\{}{\}}
% Class is a collection of sets
\newmathcommand{\class}[1]{\set{#1}}
% Class of all sets and class of all inductive sets
\newmathcommand{\UniversalClass}{\mathfrak{V}}
\newmathcommand{\InductiveClass}{\mathrm{Ind}}
% Cardinality of a set
\DeclareMathOperator{\card}{card}
% Use \varnothing as notation for empty set
\disablecommand{\emptyset}
\suggestcommand{\emptyset}{Use the better looking \varnothing.}
% Some operations on sets
\let\setminus\relax
\let\complement\relax
\newmathcommand{\setminus}{\mathbin{\circleddash}}
\newmathcommand{\symdiff}{\mathbin{\triangle}}
\newmathcommand{\complement}[1]{\overline{#1}}
\DeclareMathOperator{\powerset}{\mathcal{P}}
% Functions and relations
\let\im\relax
\let\ker\relax
\DeclareMathOperator{\im}{im}   % image
\DeclareMathOperator{\dom}{dom} % domain
\DeclareMathOperator{\cod}{cod} % codomain
\DeclareMathOperator{\ran}{ran} % range (same as codomain)
\DeclareMathOperator{\ker}{ker} % kernel
\DeclareMathOperator{\field}{field} % field of relation
% Standart mappings
\DeclareMathOperator{\id}{id} % identity
\DeclareMathOperator{\ind}{\mathds{1}} % indicator function
% Injective, surjective and bijective mappings
\newmathcommand{\injto}{\hookrightarrow}
\newmathcommand{\surto}{\twoheadrightarrow}
\newmathcommand{\bijto}{\leftrightarrow}

%%%% Topology
\DeclareMathOperator{\Cl}{Cl}   % closure
\DeclareMathOperator{\Fr}{Fr}   % boundary
\DeclareMathOperator{\Int}{Int} % interior
\DeclareMathOperator{\Out}{Out} % outerior
\DeclareMathOperator{\Hom}{Hom} % Hom-sets
\newmathcommand{\Borel}{\mathcal{B}}     % Algebra of borel sets
