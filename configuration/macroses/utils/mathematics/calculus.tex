
%%%%
%% Calcus
%%%%

%%%% Commons
\DeclarePairedDelimiterX\interval[2]{\langle}{\rangle}{#1;#2} % interval
\DeclareMathOperator{\sign}{sign}   % signum function
\DeclareMathOperator{\supp}{supp}   % funtction support
\newmathcommand{\dif}{\mathop{}\!d} % Small differntial
\newmathcommand{\Dif}{\mathop{}\!D} % Big differntial

%%%% Derivative fractions
\ExplSyntaxOn
% Generic template
\NewTemplateCommand\tmpl_derivative_frac<mm>{mmm mm}{
    \IfEmptyTF{#3}{#2{#1#5#6}{#1#7#5#4}}{\PackageError{Primes are not allowed to be before \verb|od|-like commands}}
}

\TemplateInstancePIE\od\tmpl_derivative_frac<\dif, \frac>
\TemplateInstancePIE\lod\tmpl_derivative_frac<\dif, \sfrac>
\TemplateInstancePIE\tod\tmpl_derivative_frac<\dif, \tfrac>
\TemplateInstancePIE\dod\tmpl_derivative_frac<\dif, \dfrac>

\TemplateInstancePIE\pd\tmpl_derivative_frac<\partial, \frac>
\TemplateInstancePIE\lpd\tmpl_derivative_frac<\partial, \sfrac>
\TemplateInstancePIE\tpd\tmpl_derivative_frac<\partial, \tfrac>
\TemplateInstancePIE\dpd\tmpl_derivative_frac<\partial, \dfrac>
\ExplSyntaxOff


%%%% Multivariable derivatives
\let\div\relax
\DeclareMathOperator{\grad}{grad}
\DeclareMathOperator{\rot}{rot}
\DeclareMathOperator{\div}{div}
